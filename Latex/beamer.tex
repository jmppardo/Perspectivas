\documentclass{beamer}

\usepackage{xcolor}
\newcommand{\red}[1]{\color{red}#1}

\usetheme{Madrid} %This command sets the theme for the slideshow.
% There are many themes available. Check the documentation of beamer for a complete list. You can also modify them and produce your own.
% Here is a short list that you can check: default, Madrid, CambridgeUS, JuanLesPins, Goettingen, Malmoe


\begin{document}

\title[Perspective A. \& C. Mathematics]{Perspective over Applied and Computational Mathematics}
\subtitle{A slideshow using beamer}
\author{A.U. Thor}
\institute[UC3M]{Universidad Carlos III de Madrid}
\date{\today}

\maketitle %This command creates a new slide with the title

\section{First section of the presentation}

\frame{\tableofcontents} % This command creates a slide with the table of contents



\begin{frame}
  \frametitle{Frame Title}
  \framesubtitle{Frame Subtitle}
  
  Anything that you can typeset in a latex document will fit here. This includes inline math $\cos(x)$ or displayed $$\int_a^b f(x) dx$$
  \begin{enumerate}
    \item First element
    \item Second element
  \end{enumerate}
  
  
\end{frame}

\begin{frame}
  \frametitle{Frame Title}
  \framesubtitle{Frame Subtitle}
  

  I can write some more text here 

  \begin{block}{Block title}
    This serves to \alert{highlight} some content
  \end{block}
  
  And also here


\end{frame}



\section{Second section of the presentation}

\frame{\tableofcontents[currentsection]} %The option in the \tableofcontents command presents all but the current section semitransparent.


\begin{frame}

  \frametitle{How to introduce overlays I}
  \begin{itemize}
    \pause
    \item First element
    \pause
    \item Second element
  \end{itemize}


\end{frame}

\begin{frame}
  \frametitle{How to introduce overlays II}


  \textbf<2->{This line is bold only after slide 2}
  
  \uncover<4>{I am here but am no seen until the slide 4. I take my space in the slide}
  
  \alert<3->{This text gets highlighted after slide 3}
  
  \only<1>{This line is inserted only on slide 1.}
  
  \only<2>{This line is inserted only on slide 2.}
    
  \textit<2-3>{This line is slanted in the second and third slides}
\end{frame}


\end{document}