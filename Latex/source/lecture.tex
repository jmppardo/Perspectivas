% !TEX encoding = UTF-8 Unicode
\documentclass[a4paper]{article}

\usepackage[utf8]{inputenc}% these two packages are for handlinf the encodings
\usepackage[T1]{fontenc}

\usepackage{amsmath}

\begin{document}

\title{A primer on \LaTeX{}}
\author{J.M. Pérez}
\date{\today}


\maketitle

\tableofcontents

\section{The Preamble}

The preamble is the text between the definition of the class and the beginning of the document. Packages are libraries that modify the functionality of \cite{Knuth1990}\LaTeX{}. The packages inputenc and fontenc make sure that accented caracters like ö á or even ¿ (important in spanish) work as expected.

\section{Table of contents}

The \textbackslash{}tableofcontents command prints a table of contents on the page based on the sructure of the pdf.

%\tableofcontents

\section{Creating the title}

These are the commands to get the tile (see the source file)

%\title{A primer on \LaTeX{}}
%\author{J.M. Pérez}
%\date{\today}

%\maketitle

\section{Mathematics in a Document}

\subsection{The math modes and the text mode}

Inline mode

Display mode

The inline mode is for printing mathematical formulas $\sum_j^{n=1} x^j $ within the text,  like this one. This \(\lim_{x\to\infty} f(x)=7 \).
The display mode prints the formulas on a separated line: $$ \sum_j^{n=1} x^j $$ or also \[\lim_{x\to\infty} f(x)=7\]

The math modes behave differently than the text mode, $This is math mode$, spaces are ignored.

\subsection{Some environments useful for presenting equations}

\begin{equation}\label{my_equation}
\int_{-\infty}^\infty e^{-x^2} dx = \sqrt{\pi}
\end{equation}

I can now use the numbered equation to refer to it in the text, like look at the equation \ref{my_equation}


\begin{align}
  (a+b)^2 &= a^2 + 2ab + b^2\\
  &\geq 0\notag \\
  & > -3
\end{align}

\begin{align*}
  (a+b)^2 &= a^2 + 2ab + b^2\\
  &\geq 0 \\
  & > -3
\end{align*}


\subsection{Tabular math}

there are various similar environments: matrix, pmatrix, bmatrix, cases
$$
\begin{matrix}
  a & b \\
  c & d 
\end{matrix}
$$

$$
 A = % 
  \begin{bmatrix}
    a & b \\
    c & d 
  \end{bmatrix}
$$

$$
f(x) = 
  \begin{cases}
    e^{-\frac{1}{x}} &, \text{for} x>0\\
    0 &, \text{for } x\leq 0
  \end{cases}
$$

\section{Defining a bibliography with `Bibtex'}

We have referenced the bibliography previously, like here \cite{Stone1932}. With \LaTeX{} one can use an automated tool to generate the bibliography, this is called `Bibtex'.  For doing it one needs to include the two statements that can be found after this paragraph. The source document needs to be parsed several times to complete the process. This is so because first one needs to know which references in the `.bib' file are cited in the document. Then a formatted bibliography file `.bbl' is created. Then the bibliography file is read and included in the document. Finally you need to run latex again to get the references properly. In summary, whenever you add some \textbackslash{}cite references to the document you need to compile with latex, then `Bibtex' and then with \LaTeX{} two times again. 

% Different bibliography styles. plain, siam, abbrv, alpha

\bibliographystyle{siam}
\bibliography{my_bibliography.bib}




\end{document}


